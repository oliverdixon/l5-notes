% METRIC SPACES NOTES
% OLIVER DIXON, 2023

% TODO: Do we need titletoc for \titlecontents? Can't tocloft achieve the same?
% TODO: Remove common styling elements to a document class: owdnotes.

% Panopto branding information:
% https://company-32474.frontify.com/d/Nch796CKUfjy/
%   2016-brand-style-guide#/basics/logo/vector-files

\documentclass{article}

\usepackage[
    a4paper,
    top=1.5in,
    bottom=1.5in,
    left=1in,
    right=1in,
]{geometry}
\usepackage[en-GB]{datetime2}
\usepackage{fancyhdr, anyfontsize, amsthm, mathtools, amssymb,
    tcolorbox, titletoc, titlesec, lastpage, gitinfo2, tocloft, enumitem}
\usepackage[
    colorlinks,
    allcolors=blue,
]{hyperref}
\usepackage[nameinlink]{cleveref}

% BEGIN DOCUMENT METADATA

\title{Metric Spaces}
\author{Oliver Dixon}
\date{Semester I, 2023/24}
\newcommand*\internalname{metric-spaces}
\newcommand*\modulecode{MAT00051I}

% END DOCUMENT METADATA

\newcommand*\subtitle{Consolidated Lecture Notes}
\urlstyle{same}
\tcbuselibrary{skins,theorems,breakable}
\newcommand*\fclower[2]{\relax\lowercase{#1}#2}
\renewcommand*\vec{\mathbf}

\setlength\parskip{.8em}
\setlength\parindent{0pt}
\setlength\cftparskip{0pt}
\renewcommand*\baselinestretch{1.2}

\newcommand*\iffforward{\par\boxed\Longrightarrow\ }
\newcommand*\iffbackward{\par\boxed\Longleftarrow\ }

% \newtoctheorem: create a new tcolorbox-powered theorem-like environment which
% will be included as a subsection in the table of contents.
%
%   #1: display name (e.g. Definition)
%   #2: environment name (e.g. definition)
%   #3: colour of the bounding box, according to xcolor (e.g. red)
%
\makeatletter
\newcommand*\newtoctheorem[3]{
    \newtcbtheorem[number within=section]{aux:#2}{#1}{
        enhanced,
        parbox=false,
        before upper=\hspace{-3.5pt},
        colback=white, colframe=#3, colbacktitle=white, coltitle=black,
        boxed title style={size=small,colframe=white},
        fonttitle=\bfseries,
        rounded corners=all,
        toptitle=1ex, bottomtitle=1ex, top=2ex, bottom=2ex,
        titlerule=1pt,
        title={#1},
        description font=\normalfont,
        separator sign none,
        breakable
    }{#2}
    \NewDocumentEnvironment{#2}{m+m}{
        \expandafter\csname aux:#2\endcsname{##1}{##2}
        \addcontentsline{toc}{subsection}{#1 \ref*{#2:##2}: ##1}
        \ignorespaces
    }{\expandafter\csname endaux:#2\endcsname}
    \crefname{tcb@cnt@aux:#2}{\fclower{#1}\relax}{\fclower{#1}\relax s}
    \Crefname{tcb@cnt@aux:#2}{#1}{#1s}
}
\makeatother

\renewcommand*\qedsymbol{\hfill\ensuremath{\Box}}
\titlecontents{section}[0pt]{\vskip 1ex}{\bfseries}{\bfseries}{}
\titleformat{\section}[runin]{\scshape \LARGE}{}{0pt}{}[\hfill \mbox{}]
\renewcommand*\contentsname{Lecture Contents}
\renewcommand*\cfttoctitlefont{\scshape \LARGE \hfill}
\renewcommand*\cftaftertoctitle{\hfill \mbox{}}

% \lecture: start a new lecture section, with a description and Panopto link.
%
%   #1: lecture display name
%   #2: Panopto video/folder URL suffix (or empty if no recording)
%   #3: lecture summary paragraph
%   #4: date of live delivery
%
\newcommand\lecture[4]{
    \section{#1}
    \ifstrempty{#2}{\raisebox{3pt}{\itshape No recording}}{%
        \raisebox{3pt}{\itshape #4} \hspace*{8pt}
        \href{https://york.cloud.panopto.eu/Panopto/Pages/#2}
            {\fbox{\centering\includegraphics[width=12pt]{../panopto}}}
    }%
    \par #3
    \vskip.5\baselineskip
}

\newtoctheorem{Definition}{definition}{black}
\newtoctheorem{Example}{example}{blue}
\newtoctheorem{Theorem}{theorem}{red}

\creflabelformat{equation}{#2#1#3}
\crefname{equation}{equation}{equations}
\Crefname{equation}{Equation}{Equations}
\numberwithin{equation}{section}

\setlist{itemindent=1em}
\newlist{axioms}{enumerate}{1}
\crefname{axiomsi}{axiom}{axioms}
\Crefname{axiomsi}{Axiom}{Axioms}
\newcommand*\setaxiomprefix[1]{
    \setlist[axioms]{label=#1\arabic*), ref=#1\arabic*}
}

\let\Sectionmark\sectionmark
\def\sectionmark#1{\def\sectionname{#1}\Sectionmark{#1}}
\renewcommand*\headrulewidth{0pt}
\renewcommand*\footrulewidth{\headrulewidth}
\makeatletter
\fancypagestyle{mainbody}{
	\fancyhf{}
    \fancyhead[L]{\itshape \@title: \subtitle}
    \fancyhead[R]{\itshape \@date}
    \fancyfoot[L]{\itshape \@author}
    \fancyfoot[R]{\itshape Page \thepage\ of \pageref*{LastPage}}
}
\makeatother

\begin{document}
\thispagestyle{empty}
\pagestyle{plain}
\pagenumbering{roman}
\begin{titlepage}
    \begin{flushright}
        \makeatletter
        \begingroup
            \fontsize{50}{50}\selectfont
            \slshape \sffamily \@title
            \LARGE
            \vskip\baselineskip
            \subtitle
        \endgroup
        \vfill
        \begingroup
            \Large \obeylines
            \setlength\parskip{.5em}
            Collated and Typeset by \@author
            Based on the %
            \href{https://www.york.ac.uk/students/studying/manage/%
                programmes/module-catalogue/module/\modulecode/}{\modulecode} %
            Lecture Series
            \vskip\baselineskip
            University of York
            \@date
        \endgroup
        \makeatother
    \end{flushright}
    \vfill
    \begin{center}
        \begin{tabular}{r|l}
            Compilation Date & \today \\
            Author Contact & \href{mailto: Oliver Dixon <od641@york.ac.uk>}%
                {od641@york.ac.uk} \\
            Sources Link & \url{https://github.com/oliverdixon/l5-notes/tree/%
                master/\internalname} \\
            Latest Commit Hash & \href{https://github.com/oliverdixon/l5-notes/%
                commit/\gitHash}{\gitHash} \\
            York Web Link & \url{https://www-users.york.ac.uk/~od641/l5-notes/%
                \internalname .pdf} \\
            Document Licence & \href{https://github.com/oliverdixon/l5-notes/%
                blob/master/LICENSE}{MIT Licence}
        \end{tabular}
    \end{center}
\end{titlepage}
\stepcounter{page}
\tableofcontents
\clearpage
\pagenumbering{arabic}
\pagestyle{mainbody}

\lecture{Lecture I}{Viewer.aspx?id=c3a52a78-486e-4e1d-88fc-b083009b9766}{
    Lecture One introduces the concept of a \emph{metric} as a generalisation of
    the notion of distance between two points in a set. Three \emph{canonical
    metrics} on $\mathbb{R}^N$ are presented; these are then generalised
    further, and a short proof verifies the compliance of the generalised
    Euclidean metric with the relevant axioms.
}{\DTMdisplaydate{2023}{09}{26}{Tuesday}}

\begin{definition}{Metric Space}{metric}
    Suppose that $ X $ is a set, and $ d \colon X \times X \to [0, \infty)
    \subset \mathbb{R} $. Then, $ d $ is a \emph{metric} on $ X $ if and only if
    the following properties hold for $ a, b, c \in X $:
    \setaxiomprefix{M}
    \begin{axioms}
        \item \emph{Positivity.} $ d(a, b) \geq 0 $;
        \item \emph{Equality.} $ d(a, b) = 0 \iff a = b $;
        \item \emph{Symmetry.} $ d(a, b) = d(b, a) $;
        \item \emph{Triangularity.} $ d(a, b) \leq d(a, c) + d(b, c) $
            \label{axiom:triangle-inequality}.
    \end{axioms}
    The tuple $ (X, d) $ is a \emph{metric space}.
\end{definition}
\begin{definition}{Canonical Metrics on \texorpdfstring{$\mathbb{R}^N$}{an
        N-dimensional real vector space}}{canon-metrics}
    We can consider three metrics on $ \mathbb{R}^N $: $ d_1 $, $ d_2 $, and
    $ d_\infty $, each of which have a domain of $ \mathbb{R}^N \times
    \mathbb{R}^N $ and a codomain of $ [0, \infty) $:
    \begin{align}
        d_1(\vec{x}, \vec{y}) &= \sum_{i=1}^N \vert \vec{x}_i - \vec{y}_i
            \vert \\
        d_2(\vec{x}, \vec{y}) &= \left[\sum_{i=1}^N (\vec{x}_i - \vec{y}_i)^2
            \right]^{1/2} \label{eqn:d2-metric} \\[.8em]
        d_\infty(\vec{x}, \vec{y}) &= \max_{1 \leq i \leq N} \vert \vec{x}_i -
            \vec{y}_i \vert
    \end{align}
\end{definition}
\begin{theorem}{The Generalised Metric is a Metric}{general-dp-metric}
    Consider the $ d_p $ metric, where $ d_p \colon \mathbb{R}^N \times
    \mathbb{R}^N \to [0, \infty) $ is a generalisation of the $ d_2 $ Euclidean
    metric for $ p \in \mathbb{N} $:
    \begin{equation}
        (\vec{x}, \vec{y}) \mapsto \left[\sum_{i=1}^N \vert \vec{x}_i -
        \vec{y}_i \vert^p\right]^{1/p} \label{eqn:dp-metric} \text { for all }
        \vec{x}, \vec{y} \in \mathbb{R}^N.
    \end{equation}
    Then, $ (\mathbb{R}^N, d_p) $ is a metric space.
    \begin{proof}
        To show that $ d_p $ is a metric on $ \mathbb{R}^N $, we must verify
        that $ d_p $ is in compliance with the constraints enumerated in
        \cref{definition:metric}. The positivity, equality, and symmetry axioms
        are easy to show, so we will focus on the triangularity property here,
        proving it by demonstrating a reduction to \emph{Minkowski's Theorem}.

        Let $ \vec{a}_k $ and $ \vec{b}_k $ be such that
        \begin{align}
            d_p(\vec{x}, \vec{z}) &\eqcolon \left[\sum_{k=1}^N \vert \vec{a}_k
                \vert^p\right]^{1/p} \\
            d_p(\vec{y}, \vec{z}) &\eqcolon \left[\sum_{k=1}^N \vert \vec{b}_k
                \vert^p\right]^{1/p}.
        \end{align}
        Then, note that $ d_p(\vec{x}, \vec{y}) $ (as defined in
        \cref{eqn:dp-metric}) can be
        written in terms of $ \vec{a}_k $ and $ \vec{b}_k $, since $ \vec{a}_k =
        \vec{x}_k - \vec{z}_k $ and $ \vec{b}_k = \vec{y}_k - \vec{z}_k $ for
        $ k = 1, \ldots, N $:
        \begin{equation}
            d_p(\vec{x}, \vec{y}) = \left[\sum_{k=1}^N \vert \vec{a}_k +
            \vec{b}_k \vert^p\right]^{1/p}.
        \end{equation}
        The triangle inequality, as stated in \cref{axiom:triangle-inequality},
        requires that
        \begin{equation}
            \left[\sum_{k=1}^N \vert \vec{a}_k + \vec{b}_k \vert^p\right]^{1/p}
            \leq \left[\sum_{k=1}^N \vert \vec{a}_k \vert^p\right]^{1/p} +
            \left[\sum_{k=1}^N \vert \vec{b}_k \vert^p\right]^{1/p}.
        \end{equation}
        This inequality is equivalent to the well-known Minkowski's Theorem;
        thus, $ d_p $ satisfies the triangle inequality over $ \vec{x}, \vec{y},
        \vec{z} \in \mathbb{R}^N $.
    \end{proof}
\end{theorem}

\lecture{Lecture II}{Viewer.aspx?id=e35adcae-8b46-4472-8b19-b085009bdfe1}{
    Lecture Two introduces the concept of the \emph{supremum} and \emph{infimum}
    as properties of any subset of the reals. The sets of \emph{bounded} and
    \emph{continous} functions are introduced as $ B([0, 1]) $ and $ C([0, 1]) $
    respectively, and we prove that the ``sup-metric'' $ d_\infty $ forms a
    metric on $ B([0, 1]) $.
}{\DTMdisplaydate{2023}{09}{28}{Thursday}}

\begin{definition}{Supremum and Infimum}{sup-inf}
    If $ S \subset \mathbb{R} $ is a set, then $ \sup S $ is defined to be the
    \emph{least upper bound} of $ S $. This is defined to be the smallest $ b
    \in \mathbb{R} $ such that $ x \leq b $ for all $ x \in S $. The infimum of
    $ S $, $ \inf S $, is defined analogously as the \emph{greatest lower
    bound}.
\end{definition}
\begin{definition}{The \texorpdfstring{$ \ell^\infty $}{Ell-Infinity}
        Set of Bounded Sequences}{ell-infinity}
    Consider $ \mathbb{R}^\mathbb{N} $: the set of all sequences of reals. We
    cannot work with this entire space, since many real sequences are unbounded,
    and the $ d_1 $ and $ d_2 $ canonical metrics give rise to non-finite sums.
    Therefore, we consider the set $ \ell^\infty $ as the \emph{set of
    all bounded real sequences}:
    \begin{equation}
        X \in \ell^\infty \iff \exists M > 0 \text{ such that }
            \vert X_n \vert \leq M \text { for all } n \in \mathbb{N}.
    \end{equation}
    Then, the infinity metric is defined in terms of the supremum, since a
    sequence with infinite terms mightn't possess a maximum:
    \begin{equation}
        d_\infty(X, Y) = \sup\left\{\vert X_i - Y_i \vert \colon
            i \in \mathbb{N}\right\}
    \end{equation}
    for $ X, Y \in \ell^\infty $.
\end{definition}
\begin{definition}{The Set of Bounded Functions}{bounded-functions-set}
    $ B([0,1]) $ is the \emph{set of all bounded functions} $ f $ such that
    $ f \colon [0, 1] \to \mathbb{R} $.
\end{definition}
\begin{definition}{The Set of Continuous Functions}{cts-functions-set}
    $ C([0,1]) $ is the \emph{set of all continuous functions} $ f $ such that
    $ f \colon [0, 1] \to \mathbb{R} $.
\end{definition}
\begin{theorem}{The set of bounded functions\texorpdfstring{ over $[0, 1]$}{}
    with the sup-metric forms a metric space}{bounded-is-metric}
    Consider the $ d_\infty $ metric on $ B([0, 1]) $ defined in terms of the
    supremum, such that the upper bound needn't lie in the set:
    \begin{equation}
        d_\infty \colon B([0, 1]) \times B([0, 1]) \to [0, \infty)
            \text{ such that } (f, g) \mapsto \sup\left\{\vert f(t) - g(t) \vert
            \colon t \in [0, 1] \right\}.
    \end{equation}
    Then, $ \left(B([0, 1]),\, d_\infty\right) $ is a metric space.
    \begin{proof}
        We must verify that $ d_\infty : B([0, 1]) \times B([0, 1]) \to [0,
        \infty) $ satisfies the metric axioms described in
        \cref{definition:metric} for all $ f, g, h \in B([0, 1]) $.
        \begin{itemize}
            \item Since $ f-g $ is a bounded function, there exists an $ M \geq
                0 $ for which $ f(t) - g(t) \leq M $ for all $ t \in [0, 1] $.
                Thus, $ \sup\left\{\vert f(t) - g(t) \vert \colon t \in [0,
                1]\right\} \geq 0 $, and $ d_\infty(f, g) \geq 0 $ for all $ f,
                g \in B([0, 1]) $.
            \item \iffforward If $ f = g $, then $ \vert f(t) - g(t) \vert = 0 $
                for all $ t \in [0, 1] $, so $ d_\infty(f, g) = \sup\{0, 0,
                \ldots\} = 0 $.

                \iffbackward Furthermore, if $ d_\infty(f, g) = 0 $, then we
                know that $ \sup\left\{ \vert f(t) - g(t) \vert \colon t \in [0,
                1] \right\} = 0 $. We know that $ \vert f(t) - g(t) \vert \geq 0
                $, so $ \vert f(t) - g(t) \vert = 0 $ follows immediately, from
                which we can conclude that $ f(t) = g(t) $ for all $ t \in [0,
                1] $, hence $ f = g $.

                Thus, $ d_\infty(f, g) = 0 \iff f = g $.
            \item By the symmetry of the standard metric on $ \mathbb{R} $, the
                symmetry of $ d_\infty $ on $ B([0, 1]) $ follows immediately:
                \begin{align}
                    d_\infty(f, g) &= \sup\left\{ \vert f(t) - g(t) \vert \colon
                        t \in [0, 1] \right\} \\
                    &= \sup\left\{ \vert g(t) - f(t) \vert \colon t \in [0, 1]
                        \right\} \\
                    &= d_\infty(g, f).
                \end{align}
            \item By the triangularity property of the standard metric on $
                    \mathbb{R} $,
                \begin{align}
                    d_\infty(f, g) &= \sup\left\{ \vert f(t) - g(t) \vert \colon
                        t \in [0, 1] \right\} \\
                    &= \sup\left\{ \vert f(t) - h(t) + h(t) - g(t) \vert \colon
                        t \in [0, 1] \right\} \\
                    &\leq \sup\left\{ \vert f(t) - h(t) \vert \colon t \in [0,
                        1] \right\} + \sup\left\{ \vert h(t) - g(t) \vert \colon
                        t \in [0, 1] \right\} \\
                    &= d_\infty(f, h) + d_\infty(h, g),
                \end{align}
                hence $ d_\infty $ possesses the property of triangularity on $
                B([0, 1]) $.
        \end{itemize}
        Thus, $ \left(B([0,1]), d_\infty\right) $ is a metric space.
    \end{proof}
\end{theorem}

\lecture{Lecture III}{Viewer.aspx?id=afbe7e12-8494-4fe3-b73f-b08a009bdc46}{
    Lecture Three introduces the concept of \emph{norms} as generalisations of
    the absolute value function, \emph{metric subspaces}, and \emph{isometric
    maps}, complemented by a simple example.
}{\DTMdisplaydate{2023}{10}{03}{Tuesday}}

\begin{definition}{Norm}{norm}
    A \emph{norm} is an abstraction of the absolute value. Suppose that $ V $ is
    a normable vector space. Then $ \vert\vert\cdot\vert\vert \colon V \to
    \mathbb{R} $ is such that, for all $ \vec{x}, \vec{y} \in V $,
    \setaxiomprefix{N}
    \begin{axioms}
        \item $ \vert\vert \vec{x} \vert\vert \geq 0 $;
        \item $ \vert\vert \vec{x} \vert\vert = 0 \iff \vec{x} = \vec{0} $;
        \item $ \vert\vert \lambda \vec{x} \vert\vert = \vert \lambda \vert
            \vert\vert \vec{x} \vert\vert $ for all $ \lambda \in \mathbb{R} $;
        \item $ \vert\vert \vec{x} + \vec{y} \vert\vert \leq \vert\vert \vec{x}
            \vert\vert + \vert\vert \vec{y} \vert\vert $.
    \end{axioms}
    $ V $ equipped with $ \vert\vert \cdot \vert\vert $ is a \emph{normed
    space}. Note that any norm can give rise to a metric; such a metric is
    sometimes called the \emph{metric induced by the norm}.
\end{definition}
\begin{definition}{Metric Subspace}{metric-subspace}
    Suppose that $ (X, d) $ and $ (Y, e) $ are metric spaces. We say that $ X $
    is a \emph{metric subspace} of $ Y $ if $ X \subseteq Y $ and $ d $ is a
    restriction of $ e $ to $ X \times X $.
\end{definition}
\begin{definition}{Isometric Map}{isometric-map}
    Suppose that $ (X, d) $ and $ (Y, e) $ are metric spaces, and that $ \phi
    \colon X \to Y $ is surjective. Then $ \phi $ is called an \emph{isometric
    map} if and only if $ e(\phi(a), \phi(b)) = d(a, b) $ for all $ a, b \in X
    $. This will later be used to define the most rigid definition of
    ``sameness'' for metric spaces.
\end{definition}
\begin{example}{Complex Isometry}{complex-isometry}
    Each $ (a, b) \in \mathbb{R} $ is associated with a unique $ z = a +
    \mathrm{i}b \in \mathbb{C} $ such that $ \Re(z) = a $ and $ \Im(z) = b $.
    Notice that $ (a, b) \mapsto a + \mathrm{i}b $ is a bijective map from $
    \mathbb{R}^2 $ to $ \mathbb{C} $, and hence qualifies as an isometry.
\end{example}

\lecture{Lecture IV}{Viewer.aspx?id=3a47174e-ed9d-413d-b309-b08c009bdb8e}{
    Lecture Four begins an investigation into the topology of a metric space. In
    particular, we introduce \emph{open and closed balls}, \emph{interior
    points}, \emph{boundary points}, and \emph{exterior points}, and define
    \emph{open} and \emph{closed} sets in terms of these concepts. We take an
    example metric space and rigorously compute its interior, boundary, and
    exterior, and finally show (by example) that considering objects as either
    subsets or subspaces can vastly alter their topological properties.
}{\DTMdisplaydate{2023}{10}{05}{Thursday}}

\begin{definition}{Open and Closed Balls}{balls}
    Suppose that $ (X, d) $ is a metric space, and that $ x_0 \in X $. For every
    $ \epsilon > 0 $, we define the \emph{open ball centered at $ x_0 $ with
    radius $ \epsilon $} to be the set $ B(x_0, \epsilon) = \left\{ x \in X
    \colon d(x, x_0) < \epsilon \right\} $. Analogously, the \emph{closed ball
    centered at $ x_0 $ with radius $ \epsilon $} is defined to be the set
    $ \overline{B}(x_0, \epsilon) = \left\{ x \in X \colon d(x, x_0) \leq
    \epsilon \right\} $.
\end{definition}
\begin{definition}{Interior Points}{interior-points}
    Let $ A \subset X $. An \emph{interior point} $ y \in X $ of $ A $ is an
    element for which $ B(y, \epsilon) \subset A $ for some $ \epsilon > 0 $.
    That is, there is an open ball centred at $ y $ with radius $ \epsilon $
    that is completely contained within $ A $. The set of all such points is
    denoted as $ A^o $, and is called \emph{the interior of $ A $}.
\end{definition}
\begin{definition}{Boundary Points}{boundary-points}
    The element $ y \in X $ is a \emph{boundary point} of $ A $ if and only if
    for any $ \epsilon > 0 $, $ B(y, \epsilon) \cap A \neq \emptyset $ and
    $ B(y, \epsilon) \cap A^c \neq \emptyset $. That is, any open ball centred
    at $ y $ always intersects with $ A $ and its complement $ A^c $. The set of
    all such points is denoted as $ \partial A $, and is called \emph{the
    boundary of $ A $}.
\end{definition}
\begin{definition}{Exterior Points}{exterior-points}
    The element $ y \in X $ is an \emph{exterior point} of A if and only if for
    some $ \epsilon > 0 $, $ B(y, \epsilon) \subset A^c $. That is, there exists
    an open ball centred at $ y $ which intersects only with the complement of
    $ A $; this can also be interpreted as an interior point of the complement.
    The set of all such points is denoted as $ A^e $, and is called \emph{the
    exterior of A}.
\end{definition}
\begin{example}{Finding the interior, boundary, and exterior of a
        set}{int-bound-ext}
    Consider $ (\mathbb{R}, d) $ with $ A = (0, 1] \subset \mathbb{R} $.
    Intuitively, we can conjecture that $ A^o = (0, 1) $, $ \partial A = \{0,
    1\} $, and $ A^e = \mathbb{R} \setminus (0, 1) = (-\infty, 0) \cup (1,
    \infty) $, however these claims must be proven rigorously by (a) showing
    that the conjectured points do belong to the relevant set, and (b) showing
    that the conjectured points are the only elements to belong to the relevant
    set.
    \begin{itemize}
        \item First consider the interior. Take $ x \in (0, 1) $. By
            \cref{definition:interior-points}, we want to show that there is an
            $ \epsilon > 0 $ such that $ B(x, \epsilon) \subset (0, 1] = A $.
            Set $ \epsilon_1 \coloneq x $, and $ \epsilon_2 \coloneq 1-x $.
            Given that $ 0 < x < 1 $, we can take an $ \epsilon \coloneq
            \min\{ \epsilon_1, \epsilon_2 \} $. Then, since $ \epsilon/2 <
            \epsilon_1, \epsilon_2 $,
            \begin{equation}
                B(x, \epsilon/2) = \left\{ y \in \mathbb{R} \colon d(x, y) <
                    \epsilon/2 \right\} \subset A.
            \end{equation}
            This proves that $ (0, 1) \subseteq A^o $.

            We now need to eliminate the remaining candidates in $ \mathbb{R}
            \setminus (0, 1) $ from having possible membership in $ A^o $. The
            points $ x < 0 $ and $ x > 1 $ can be discarded immediately, since
            any open ball centred at these points could never lie totally within
            $ A $, due to their positive radii $ \epsilon $. Finally, we need to
            show that $ \{0,1\} \not\subset A^o $. Without loss of generality,
            pick $ x = 1 $, and take an $ \epsilon > 0 $ to consider the open
            ball $ B(x, \epsilon) $. Any such ball would contain a point that is
            strictly greater than 1, and hence would contain points outside of $
            A $. Thus, $ x \not\in A^o $, and $ A^o = (0, 1) $ as claimed.
        \item Now consider the boundary, as described in
            \cref{definition:boundary-points}. We claim that $ \{0, 1\} =
            \partial A $, and first demonstrate that $ \{0, 1\} \subseteq
            \partial A $.  Without loss of generality, we show that $ 0 \in
            \partial A $. Let $ \epsilon > 0 $, and consider $ B(0, \epsilon) =
            (-\epsilon, \epsilon) $. Clearly, since $ \epsilon > 0 $, $
            (-\epsilon, \epsilon) \cap A \neq \emptyset $ and $ (-\epsilon,
            \epsilon) \cap A^c \neq \emptyset $; thus, $ 0 $ is a boundary
            point.

            Now, we show that there are no other boundary points of $ A $ in $
            \mathbb{R} $. We know that $ \mathbb{R} = A^o \coprod \partial A
            \coprod A^e $, hence $ A^o \cup \partial A = \emptyset $, thus $ (0,
            1) \not\subset \partial A $. Without loss of generality for $ x < 0
            $, consider points $ x > 1 $. Therefore, there exists an $ \epsilon
            > 0 $ such that $ x = 1 + \epsilon $. Considering $ B(x, \epsilon/2)
            $, we can see that $ B(x, \epsilon/2) \subset A^c $, which implies
            that $ B(x, \epsilon/2) \cap A = \emptyset $. Thus, $ \{ 0, 1 \} =
            \partial A $.
        \item Finally, consider the exterior, as described in
            \cref{definition:exterior-points}. Recall that the entire space can
            be expressed as a disjoint union, e.g. $ \mathbb{R} = A^o \coprod
            A^e \coprod \partial A $. Hence,
            \begin{align}
                A^e &= \mathbb{R} \setminus \left( A^o \cup \partial A
                    \right) \\
                &= \mathbb{R} \setminus [0, 1] \\
                &= (-\infty, 0) \cup (1, \infty),
            \end{align}
            as conjectured. $ \qedsymbol $
    \end{itemize}
\end{example}
\begin{definition}{Open, Closed, and Clopen Sets}{open-closed-clopen-sets}
    Let $ (X, d) $ be a metric space. A subset $ A $ of $ X $ is \emph{open} if
    and only if $ A \cap \partial A = \emptyset $. A subset $ F $ of $ X $ is
    \emph{closed} if and only if $ \partial F \subseteq F $. Note that a set can
    be both open and closed: typical examples are the empty set and the entire
    space; these sets are called \emph{clopen}.
\end{definition}
\begin{example}{Subset vs. Subspace}{subset-subspace}
    Consider $ (\mathbb{R}, d) $ with $ A = (0, 1) \cup (1, 2) $. If $ A $ is
    considered as a \emph{subset} if $ \mathbb{R} $, then $ \partial A = \{0, 1,
    2 \} $. Hence, $ A \cap \partial A = \emptyset $, and $ A $ is open by
    \cref{definition:open-closed-clopen-sets}.

    Further, $ A $ is not closed, since $ \partial A \not\subseteq A $. If $ A $
    is considered as a \emph{subspace} of $ \mathbb{R} $, then $ \partial A =
    \emptyset $, since $ \{ 0, 1, 2 \} \not\subseteq A $, and we cannot consider
    the points outside of the subspace when determining its topology. Hence, $ A
    $ is closed. But $ A \cap \partial A = \emptyset $, since $ \partial A =
    \emptyset $; thus $ A $ is also open, and is ultimately clopen when
    interpreted as a subspace.
\end{example}

\lecture{Lecture V}{Viewer.aspx?id=dbcb156d-d828-454f-929d-b091009bf65b}{
    TODO Description
}{\DTMdisplaydate{2023}{10}{10}{Tuesday}}

\begin{theorem}{A subset is open if and only if its complement is
        closed}{open-subset-closed-complement}
    Consider a set $ A \subseteq X $. Then, $ A $ is open if and only if $ A^c $
    is closed.
    \begin{proof}
        This can be proven by unravelling the definitions of open and closed
        sets (\cref{definition:open-closed-clopen-sets}) and boundary points
        (\cref{definition:boundary-points}).

        \iffforward First, suppose that $ A $ is open.  If $ A = \emptyset $,
        then $ A^c = X $. The entire space is known to be clopen, and thus
        closed. If $ A \neq \emptyset $, then $ A \cap \partial A = \emptyset $,
        and $ \partial A \subseteq A^c $.  We can now see a useful equality by
        using the fact that $ \left(A^c\right)^c = A $:
        \begin{align}
            \partial \left(A^c\right) &= \left\{ y \in X \colon \forall \epsilon
                > 0 \, B(y, \epsilon) \cap A^c \neq \emptyset \land B(x,
                \epsilon) \cap \left(A^c\right)^c \neq \emptyset \right\} \\
            &= \left\{ y \in X \colon \forall \epsilon > 0 \, B(y, \epsilon)
                \cap A^c \neq \emptyset \land B(x, \epsilon) \cap A \neq
                \emptyset \right\} \\
            &= \partial A.
        \end{align}
        Since $ \partial A \subseteq A^c $, and $ \partial A = \partial
        \left(A^c\right) $, we know that $ \partial \left(A^c\right) \subseteq
        A^c $. Hence, $ A^c $ is closed.

        \iffbackward Next, assume that $ A^c $ is closed, hence $ \partial
        \left(A^c\right) \subseteq A^c $. Given that $ \partial A = \partial
        \left(A^c\right) $, $ \partial A \subseteq A^c $. Since $ A^c \cap A =
        \emptyset $, we know that $ \partial A \cap A = \emptyset $, and thus $
        A $ is open.
    \end{proof}
\end{theorem}
\begin{definition}{The topology of a metric space}{topology}
    The \emph{topology} of a metric space $ (X, d) $ is denoted as $ T_d $, and
    is defined to be \emph{the collection of all open subsets of $ X $}. Note
    that since $ T_d \subseteq \mathcal{P}(X) $, for any set $ X $, $ \emptyset,
    X \in T_d $, so $ T_d \neq \emptyset $.
\end{definition}
\begin{theorem}{Equivalence between openness and the existence of open balls}
        {openness-open-balls}
    Let $ A \subseteq X $ be open. Then, every point of $ A $ is an interior
    point of A. Equivalently, $ A $ is open if and only if there is an open ball
    around every point in $ A $ that resides within $ A $:
    \begin{equation}
        \forall x \in A\, \exists \epsilon > 0 \text{ such that } B (x,
        \epsilon) \subseteq A.
    \end{equation}

    \begin{proof}
        \iffforward First assume that $ A $ is open. By definition, $ A \cap
        \partial A = \emptyset $. If $ A = \emptyset $, then there exists no
        points to select, and the universal quantifier cannot select any points
        for $ x $. If $ A \neq \emptyset $, then there must be at least one $
        \epsilon > 0 $ such that $ B(x, \epsilon) \subseteq A $ or $ B(x,
        \epsilon) \subseteq A^c $ for any $ x \in A $. But, since $ x \in A $,
        it is not possible that $ B(x, \epsilon) $ is entirely contained within
        $ A^c $, since $ x \in B(x, \epsilon) $. Hence, $ B(x, \epsilon)
        \subseteq A^c $, as required.

        \iffbackward Next, suppose that $ \forall x \in A\, \exists \epsilon > 0
        $ such that $ B (x, \epsilon) \subseteq A $. Take any $ x \in A $.
        Immediately, we can see that $ x \not\in \partial A $, since $ B(x,
        \epsilon) \cap A^c = \emptyset $, because $ B(x, \epsilon) \subseteq A
        $. Hence, $ A $ is open.
    \end{proof}
\end{theorem}

\end{document}

