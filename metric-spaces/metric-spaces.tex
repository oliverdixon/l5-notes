% METRIC SPACES NOTES
% OLIVER DIXON, 2023

\documentclass{article}

\usepackage[a4paper]{geometry}
\usepackage[en-GB]{datetime2}
\usepackage{fancyhdr, anyfontsize, amsthm, mathtools, amssymb, thmtools,
    tcolorbox, xparse, titletoc, titlesec, lastpage}
\usepackage[dirty=\ (dirty)]{gitinfo2}
\usepackage[
    colorlinks,
    allcolors = blue
]{hyperref}
\usepackage[nameinlink]{cleveref}

% BEGIN DOCUMENT METADATA

\title{Metric Spaces}
\author{Oliver Dixon}
\date{Semester 1, 2023/24}
\newcommand*\internalname{metric-spaces}
\newcommand*\modulecode{MAT00051I}

% END DOCUMENT METADATA

\urlstyle{same}
\tcbuselibrary{skins,theorems,breakable}
\newcommand*\fclower[2]{\relax\lowercase{#1}#2}
\renewcommand*\vec{\mathbf}

% \newtoctheorem: create a new tcolorbox-powered theorem-like environment which
% will be included as a subsection in the table of contents.
%
%   #1: display name (e.g. Definition)
%   #2: environment name (e.g. definition)
%   #3: colour of the bounding box, according to xcolor (e.g. red)
%
\makeatletter
\newcommand*\newtoctheorem[3]{
    \newtcbtheorem[number within=section]{aux:#2}{#1}{
        enhanced,
        colback=white, colframe=#3, colbacktitle=white, coltitle=black,
        boxed title style={size=small,colframe=white},
        fonttitle=\bfseries,
        rounded corners=all,
        toptitle=1ex, bottomtitle=1ex, top=2ex, bottom=2ex,
        titlerule=1pt,
        title={#1},
        description font=\normalfont,
        separator sign none,
        breakable
    }{#2}
    \NewDocumentEnvironment{#2}{m+m}
        {
            \expandafter\csname aux:#2\endcsname{##1}{##2}
            \addcontentsline{toc}{subsection}{#1 \ref{#2:##2}: ##1}
            \ignorespaces
        }
        {\expandafter\csname endaux:#2\endcsname}
    \crefname{tcb@cnt@aux:#2}{\fclower{#1}\relax}{\fclower{#1}\relax s}
    \Crefname{tcb@cnt@aux:#2}{#1}{#1s}
}
\makeatother

\renewcommand*\qedsymbol{\hfill\ensuremath{\Box}}
\renewcommand*\contentsname{\thispagestyle{plain} Lecture Summaries}
\titlecontents{section}[0pt]{\vskip 1ex}{\bfseries}{\bfseries}{}
\titleformat{\section}{
    \centering\scshape\LARGE}
    {}{0pt}{}

\newtoctheorem{Definition}{definition}{black}
\newtoctheorem{Example}{example}{blue}
\newtoctheorem{Theorem}{theorem}{red}

\newtagform{noparen}{}{}
\usetagform{noparen}
\creflabelformat{equation}{#2#1#3}
\crefname{equation}{equation}{equations}
\Crefname{equation}{Equation}{Equations}
\numberwithin{equation}{section}
\numberwithin{enumi}{section}

\let\Sectionmark\sectionmark
\def\sectionmark#1{\def\sectionname{#1}\Sectionmark{#1}}
\renewcommand*\headrulewidth{0pt}
\renewcommand*\footrulewidth{\headrulewidth}
\makeatletter
\fancypagestyle{mainbody}{
	\fancyhf{}
    \fancyhead[L]{\itshape \@title}
    \fancyhead[R]{\itshape \@date}
    \fancyfoot[L]{\itshape \@author}
    \fancyfoot[R]{\itshape Page \thepage\ of \pageref*{LastPage}}
}
\makeatother

\begin{document}
\pagenumbering{roman}
\pagestyle{empty}
\begin{titlepage}
    \newgeometry{top=1.5in, bottom=1in, right=1in, left=1in}
    \begin{flushright}
        \makeatletter
        \begingroup
            \fontsize{50}{50}\selectfont
            \slshape \sffamily \@title
        \endgroup
        \vfill
        \begingroup
            \LARGE \obeylines
            \setlength{\parskip}{.5em}
            Typeset by \@author
            Based on \href{https://www.york.ac.uk/students/studying/manage/%
                programmes/module-catalogue/module/\modulecode/}{\modulecode}
            \vskip\baselineskip
            University of York
            Semester I, 2023/24
        \endgroup
        \makeatother
    \end{flushright}
    \vfill
    \begin{center}
        \large
        \centering
        \begin{tabular}{r|l}
            Compilation Date & \today \\
            Author Contact & \href{mailto: Oliver Dixon <od641@york.ac.uk>}%
                {od641@york.ac.uk} \\
            Sources Link & \url{https://github.com/oliverdixon/l5-notes/%
                \internalname} \\
            Latest Commit Hash & \href{https://github.com/oliverdixon/l5-notes/%
                commit/\gitHash}{\gitHash}\gitDirty \\
            York Web Link & \url{https://www-users.york.ac.uk/~od641/l5-notes/%
                \internalname}
        \end{tabular}
    \end{center}
\end{titlepage}
\stepcounter{page}
\restoregeometry
\tableofcontents
\clearpage
\pagenumbering{arabic}
\pagestyle{mainbody}

\section{Lecture I}

Lecture One introduces the concept of a \emph{metric} as a generalisation of the
notion of distance between two points in a set. Three \emph{canonical metrics}
on $\mathbb{R}^N$ are presented; these are then generalised further, and a short
proof verifies the compliance of the generalised metric with the relevant
axioms.
\begin{definition}{Metric}{metric}
    Suppose that $ X $ is a set, and $ d \colon X \times X \to [0, \infty)
    \subset \mathbb{R} $. Then, $ d $ is a \emph{metric} on $ X $ if and only if
    the following properties hold for $ a, b, c \in X $:
    \begin{enumerate}
        \item \emph{Positivity.} $ d(a, b) \geq 0 $;
        \item \emph{Equality.} $ d(a, b) = 0 \iff a = b $;
        \item \emph{Symmetry.} $ d(a, b) = d(b, a) $;
        \item \emph{Triangularity.} $ d(a, b) \leq d(a, c) + d(b, c) $
            \label{itm:triangle-inequality}.
    \end{enumerate}
\end{definition}
\begin{definition}{Canonical Metrics on \texorpdfstring{$\mathbb{R}^N$}{an
        N-dimensional real vector space}}{canon-metrics}
    We can consider three metrics on $ \mathbb{R}^N $: $ d_1 $, $ d_2 $, and
    $ d_\infty $, each of which have a domain of $ \mathbb{R}^N \times
    \mathbb{R}^N $ and a codomain of $ [0, \infty) $:
    \begin{align}
        d_1(\vec{x}, \vec{y}) &= \sum_{i=1}^N \vert \vec{x}_i - \vec{y}_i
            \vert \\
        d_2(\vec{x}, \vec{y}) &= \left[\sum_{i=1}^N (\vec{x}_i - \vec{y}_i)^2
            \right]^{1/2} \label{eqn:d2-metric} \\[.8em]
        d_\infty(\vec{x}, \vec{y}) &= \max_{1 \leq i \leq N} \vert \vec{x}_i -
            \vec{y}_i \vert
    \end{align}
\end{definition}
\begin{theorem}{The Generalised Metric is a Metric}{general-dp-metric}
    We can generalise the $d_2$ metric from \cref{eqn:d2-metric} to any $ p
    \in \mathbb{N} $, such that $ d_p \colon \mathbb{R}^N \times \mathbb{R}^N
    \to [0, \infty) $, where
    \begin{equation}
        d_p(\vec{x}, \vec{y}) = \left[\sum_{i=1}^N \vert \vec{x}_i - \vec{y}_i
            \vert^p\right]^{1/p} \label{eqn:dp-metric}.
    \end{equation}
    To show that $ d_p $ is a metric on $ \mathbb{R}^N $, we must verify that $
    d_p $ is in compliance with the constraints described in
    \cref{definition:metric}. The positivity, equality, and symmetry axioms are
    easy to show, so we will focus on the triangularity property here, proving
    it by demonstrating a reduction to \emph{Minkowski's Theorem}.

    Let $ \vec{a}_k $ and $ \vec{b}_k $ be such that
    \begin{align}
        d_p(\vec{x}, \vec{z}) &\eqcolon \left[\sum_{k=1}^N \vert \vec{a}_k
            \vert^p\right]^{1/p} \\
        d_p(\vec{y}, \vec{z}) &\eqcolon \left[\sum_{k=1}^N \vert \vec{b}_k
            \vert^p\right]^{1/p}.
    \end{align}
    Then, note that $ d_p(\vec{x}, \vec{y}) $ (as defined in
    \cref{eqn:dp-metric}) can be
    written in terms of $ \vec{a}_k $ and $ \vec{b}_k $, since $ \vec{a}_k =
    \vec{x}_k - \vec{z}_k $ and $ \vec{b}_k = \vec{y}_k - \vec{z}_k $ for
    $ k = 1, \ldots, N $:
    \begin{equation}
        d_p(\vec{x}, \vec{y}) = \left[\sum_{k=1}^N \vert \vec{a}_k + \vec{b}_k
            \vert^p\right]^{1/p}.
    \end{equation}
    The triangle inequality, as stated in \cref{itm:triangle-inequality},
    requires that
    \begin{equation}
        \left[\sum_{k=1}^N \vert \vec{a}_k + \vec{b}_k \vert^p\right]^{1/p} \leq
        \left[\sum_{k=1}^N \vert \vec{a}_k \vert^p\right]^{1/p} +
        \left[\sum_{k=1}^N \vert \vec{b}_k \vert^p\right]^{1/p}.
    \end{equation}
    This inequality is equivalent to the well-known Minkowski's Theorem; thus,
    $ d_p $ satisfies the triangle inequality over $ \vec{x}, \vec{y}, \vec{z}
    \in \mathbb{R}^N $. $ \qedsymbol $
\end{theorem}

\section{Lecture II}

Lecture II introduces the concept of the \emph{supremum} and \emph{infimum} as
properties of any subset of the reals. The sets of \emph{bounded} and
\emph{continous} functions are introduced as $ B([0, 1]) $ and $ C([0, 1]) $
respectively, and we prove that the ``sup-metric'' $ d_\infty $ forms a metric
on $ B([0, 1]) $.
\begin{definition}{Supremum and Infimum}{sup-inf}
    If $ S \subset \mathbb{R} $ is a set, then $ \sup S $ is defined to be the
    \emph{least upper bound} of $ S $. This is defined to be the smallest $ b
    \in \mathbb{R} $ such that $ x \leq b $ for all $ x \in S $. The infimum of
    $ S $, $ \inf S $, is defined analogously as the \emph{greatest lower
    bound}.
\end{definition}
\begin{definition}{The \texorpdfstring{$ \ell^\infty $}{Ell-Infinity}
        Set of Bounded Sequences}{ell-infinity}
    Consider $ \mathbb{R}^\mathbb{N} $: the set of all sequences of reals. We
    cannot work with this entire space, since many real sequences are unbounded,
    and the $ d_1 $ and $ d_2 $ canonical metrics give rise to non-finite sums.
    Therefore, we consider the set $ \ell^\infty $ as the \emph{set of
    all bounded real sequences}:
    \begin{equation}
        X \in \ell^\infty \iff \exists M > 0 \text{ such that }
            \vert X_n \vert \leq M \text { for all } n \in \mathbb{N}.
    \end{equation}
    Then, the infinity metric is defined in terms of the supremum, since a
    sequence with infinite terms mightn't possess a maximum:
    \begin{equation}
        d_\infty(X, Y) = \sup\left\{\vert X_i - Y_i \vert \colon
            i \in \mathbb{N}\right\}
    \end{equation}
    for $ X, Y \in \ell^\infty $.
\end{definition}
\begin{definition}{The Set of Bounded Functions}{bounded-functions-set}
    $ B([0,1]) $ is the \emph{set of all bounded functions} $ f $ such that
    $ f \colon [0, 1] \to \mathbb{R} $.
\end{definition}
\begin{definition}{The Set of Continuous Functions}{cts-functions-set}
    $ C([0,1]) $ is the \emph{set of all continuous functions} $ f $ such that
    $ f \colon [0, 1] \to \mathbb{R} $.
\end{definition}
\begin{theorem}{The set of bounded functions with the infinity metric forms a
        metric space}{bounded-is-metric}
    We claim that $ B([0, 1],\, d_\infty) $ forms a metric space. As is routine,
    we must verify that $ d_\infty : B([0, 1]) \times B([0, 1]) \to [0, \infty)
    $ satisfies the metric axioms described in \cref{definition:metric} for all
    $ f, g, h \in B([0, 1]) $.
    \begin{itemize}
        \item Since $ f-g $ is a bounded function, there exists an $ M \geq 0 $
            for which $ f(t) - g(t) \leq M $ for all $ t \in [0, 1] $. Thus,
            $ \sup{\vert f(t) - g(t) \vert \colon t \in [0, 1]} \geq 0 $, and
            $ d_\infty(f, g) \geq 0 $ for all $ f, g \in B([0, 1]) $.
        \item If $ f = g $, then $ \vert f(t) - g(t) \vert = 0 $ for all $ t \in
            [0, 1] $, so $ d_\infty(f, g) = \sup\{0, 0, \ldots\} = 0 $.
            Furthermore, if $ d_\infty(f, g) = 0 $, then we know that $
            \sup\left\{ \vert f(t) - g(t) \vert \colon t \in [0, 1] \right\} = 0
            $. We know that $ \vert f(t) - g(t) \vert \geq 0 $, so $ \vert f(t)
            - g(t) \vert = 0 $ follows immediately, from which we can conclude
            that $ f(t) = g(t) $ for all $ t \in [0, 1] $, hence $ f = g $.
            Thus, $ d_\infty(f, g) = 0 \iff f = g $.
        \item By the symmetry of the standard metric on $ \mathbb{R} $, the
            symmetry of $ d_\infty $ on $ B([0, 1]) $ follows immediately:
            \begin{align}
                d_\infty(f, g) &= \sup\left\{ \vert f(t) - g(t) \vert \colon
                    t \in [0, 1] \right\} \\
                &= \sup\left\{ \vert g(t) - f(t) \vert \colon t \in [0, 1]
                    \right\} \\
                &= d_\infty(g, f).
            \end{align}
        \item By the triangle inequality on $ \mathbb{R} $,
            \begin{align}
                d_\infty(f, g) &= \sup\left\{ \vert f(t) - g(t) \vert \colon
                    t \in [0, 1] \right\} \\
                &= \sup\left\{ \vert f(t) - h(t) + h(t) - g(t) \vert \colon
                    t \in [0, 1] \right\} \\
                &\leq \sup\left\{ \vert f(t) - h(t) \vert \colon t \in [0, 1]
                    \right\} + \sup\left\{ \vert h(t) - g(t) \vert \colon t \in
                    [0, 1] \right\} \\
                &= d_\infty(f, h) + d_\infty(h, g),
            \end{align}
            hence $ d_\infty $ possesses the property of triangularity on $
            B([0, 1]) $. $ \qedsymbol $
    \end{itemize}
\end{theorem}

\section{Lecture III}

Lecture Three introduces the concept of \emph{norms}, \emph{metric subspaces},
and \emph{isometric maps}, complemented by a simple example.
\begin{definition}{Norm}{norm}
    A \emph{norm} is an abstraction of the absolute value. Suppose that $ V $ is
    a normable vector space. Then $ \vert\vert\cdot\vert\vert \colon V \to
    \mathbb{R} $ is such that, for all $ \vec{x}, \vec{y} \in V $,
    \begin{enumerate}
        \item $ \vert\vert \vec{x} \vert\vert \geq 0 $;
        \item $ \vert\vert \vec{x} \vert\vert = 0 \iff \vec{x} = \vec{0} $;
        \item $ \vert\vert \lambda \vec{x} \vert\vert = \vert \lambda \vert
            \vert\vert \vec{x} \vert\vert $ for all $ \lambda \in \mathbb{R} $;
        \item $ \vert\vert \vec{x} + \vec{y} \vert\vert \leq \vert\vert \vec{x}
            \vert\vert + \vert\vert \vec{y} \vert\vert $.
    \end{enumerate}
    $ V $ equipped with $ \vert\vert \cdot \vert\vert $ is a \emph{normed
    space}. Note that any norm can give rise to a metric; such a metric is
    sometimes called the \emph{metric induced by the norm}.
\end{definition}
\begin{definition}{Metric Subspace}{metric-subspace}
    Suppose that $ (X, d) $ and $ (Y, e) $ are metric spaces. We say that $ X $
    is a \emph{metric subspace} of $ Y $ if $ X \subseteq Y $ and $ d $ is a
    restriction of $ e $ to $ X \times X $.
\end{definition}
\begin{definition}{Isometric Map}{isometric-map}
    Suppose that $ (X, d) $ and $ (Y, e) $ are metric spaces, and that $ \phi
    \colon X \to Y $ is surjective. Then $ \phi $ is called an \emph{isometric
    map} if and only if $ e(\phi(a), \phi(b)) = d(a, b) $ for all $ a, b \in X
    $. This will later be used to define the most rigid definition of
    ``sameness'' for metric spaces.
\end{definition}
\begin{example}{Complex Isometry}{complex-isometry}
    Each $ (a, b) \in \mathbb{R} $ is associated with a unique $ z = a +
    \mathrm{i}b \in \mathbb{C} $ such that $ \Re(z) = a $ and $ \Im(z) = b $.
    Notice that $ (a, b) \mapsto a + \mathrm{i}b $ is a bijective map from $
    \mathbb{R}^2 $ to $ \mathbb{C} $, and hence qualifies as an isometry.
\end{example}

\end{document}

