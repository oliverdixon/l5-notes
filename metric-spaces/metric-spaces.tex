% METRIC SPACES NOTES
% OLIVER DIXON, 2023

\documentclass{article}

\usepackage[a4paper]{geometry}
\usepackage[en-GB]{datetime2}
\usepackage{fancyhdr, anyfontsize, amsthm, mathtools, amssymb, thmtools,
    tcolorbox, xparse, titletoc, titlesec}
\usepackage[dirty=\ (dirty)]{gitinfo2}
\usepackage[
    colorlinks,
    allcolors = blue
]{hyperref}
\usepackage[nameinlink]{cleveref}

% BEGIN DOCUMENT METADATA

\title{Metric Spaces}
\author{Oliver Dixon}
\date{Semester 1, 2023/24}
\newcommand\internalname{metric-spaces}
\newcommand\modulecode{MAT00051I}

% END DOCUMENT METADATA

\tcbuselibrary{skins,theorems,breakable}
\newcommand\fclower[2]{\relax\lowercase{#1}#2}

% \newtoctheorem: create a new tcolorbox-powered theorem-like environment which
% will be included as a subsection in the table of contents.
%
%   #1: display name (e.g. Definition)
%   #2: environment name (e.g. definition)
%   #3: colour of the bounding box, according to xcolor (e.g. red)
%
\makeatletter
\newcommand\newtoctheorem[3]{
    \newtcbtheorem[number within=section]{aux:#2}{#1}{
        enhanced,
        colback=white, colframe=#3, colbacktitle=white, coltitle=black,
        boxed title style={size=small,colframe=white},
        fonttitle=\bfseries,
        rounded corners=all,
        toptitle=1ex, bottomtitle=1ex, top=2ex, bottom=2ex,
        titlerule=1pt,
        title={#1},
        description delimiters parenthesis,
        description font=\normalfont,
        separator sign none,
        breakable
    }{#2}
    \NewDocumentEnvironment{#2}{m+m}
        {
            \expandafter\csname aux:#2\endcsname{##1}{##2}
            \addcontentsline{toc}{subsection}{#1 \ref{#2:##2}: ##1}
            \ignorespaces
        }
        {\expandafter\csname endaux:#2\endcsname}
    \crefname{tcb@cnt@aux:#2}{\fclower{#1}\relax}{\fclower{#1}\relax s}
    \Crefname{tcb@cnt@aux:#2}{#1}{#1s}
}
\makeatother

\renewcommand*\contentsname{Lecture Summaries}
\titlecontents{section}[0pt]{\vskip 1ex}{\bfseries}{\bfseries}{}
\titleformat{\section}{\centering\scshape\LARGE}{}{0pt}{}

\newtoctheorem{Definition}{definition}{black}
\newtoctheorem{Example}{example}{blue}
\newtoctheorem{Theorem}{theorem}{red}

\newtagform{noparen}{}{}
\usetagform{noparen}
\creflabelformat{equation}{#2#1#3}
\crefname{equation}{equation}{equations}
\Crefname{equation}{Equation}{Equations}
\numberwithin{equation}{section}
\numberwithin{enumi}{section}

\urlstyle{same}

\begin{document}
\pagenumbering{roman}
\begin{titlepage}
    \newgeometry{top=1.5in, bottom=1in, right=1in, left=1in}
    \begin{flushright}
        \makeatletter
        \begingroup
            \fontsize{50}{50}\selectfont
            \slshape \sffamily \@title
        \endgroup
        \vfill
        \begingroup
            \LARGE \obeylines
            \setlength{\parskip}{.5em}
            Typeset by \@author
            Based on \href{https://www.york.ac.uk/students/studying/manage/%
                programmes/module-catalogue/module/\modulecode/}{\modulecode}
            \vskip\baselineskip
            University of York
            Semester I, 2023/24
        \endgroup
        \makeatother
    \end{flushright}
    \vfill
    \begin{center}
        \large
        \centering
        \begin{tabular}{r|l}
            Compilation Date & \today \\
            Author Contact & \href{mailto: Oliver Dixon <od641@york.ac.uk>}%
                {od641@york.ac.uk} \\
            Sources Link & \url{https://github.com/oliverdixon/l5-notes/%
                \internalname} \\
            Latest Commit Hash & \href{https://github.com/oliverdixon/l5-notes/%
                commit/\gitHash}{\gitHash}\gitDirty \\
            York Web Link & \url{https://www-users.york.ac.uk/~od641/l5-notes/%
                \internalname}
        \end{tabular}
    \end{center}
\end{titlepage}
\stepcounter{page}
\restoregeometry
\tableofcontents
\clearpage
\pagenumbering{arabic}
\section{Lecture I}
Lecture One introduces the concept of a \emph{metric} as a generalisation of the
notion of distance between two points in a set. Three \emph{canonical metrics}
on $\mathbb{R}^N$ are presented; these are then generalised further, and a short
proof verifies the compliance of the generalised metric with the relevant
axioms.
\begin{definition}{Metric}{metric}
    Suppose that $ X $ is a set, and $ d \colon X \times X \to [0, \infty)
    \subset \mathbb{R} $. Then, $ d $ is a \emph{metric} on $ X $ if and only if
    the following properties hold for $ a, b, c \in X $:
    \begin{enumerate}
        \item \emph{Positivity.} $ d(a, b) \geq 0 $;
        \item \emph{Equality.} $ d(a, b) = 0 \iff a = b $;
        \item \emph{Symmetry.} $ d(a, b) = d(b, a) $;
        \item \emph{Triangularity.} $ d(a, b) \leq d(a, c) + d(b, c) $
            \label{itm:triangle-inequality}.
    \end{enumerate}
\end{definition}
\begin{definition}{Canonical Metrics on \texorpdfstring{$\mathbb{R}^N$}{an
        N-dimensional real vector space}}{canon-metrics}
    We can consider three metrics on $ \mathbb{R}^N $: $ d_1 $, $ d_2 $, and
    $ d_\infty $, each of which have a domain of $ \mathbb{R}^N \times
    \mathbb{R}^N $ and a codomain of $ [0, \infty) $:
    \begin{align}
        d_1(x, y) &= \sum_{i=1}^N \vert x_i - y_i \vert \\
        d_2(x, y) &= \left[\sum_{i=1}^N (x_i - y_i)^2\right]^{1/2}
            \label{eqn:d2-metric} \\[.8em]
        d_\infty(x, y) &= \max_{1 \leq i \leq N} \vert x_i - y_i \vert
    \end{align}
\end{definition}
\begin{theorem}{The Generalised Metric is a Metric}{general-dp-metric}
    We can generalise the $d_2$ metric from \cref{eqn:d2-metric} to any $ p
    \in \mathbb{N} $, such that $ d_p \colon \mathbb{R}^N \times \mathbb{R}^N
    \to [0, \infty) $, where
    \begin{equation}
        d_p(x, y) = \left[\sum_{i=1}^N \vert x_i - y_i \vert^p\right]^{1/p}.
        \label{eqn:dp-metric}
    \end{equation}
    To show that $ d_p $ is a metric on $ \mathbb{R}^N $, we must verify that $
    d_p $ is in compliance with the constraints described in
    \cref{definition:metric}. The positivity, equality, and symmetry axioms are
    easy to show, so we will focus on the triangularity property here, proving
    it by demonstrating a reduction to \emph{Minkowski's Theorem}.

    Let $ a_k $ and $ b_k $ be such that
    \begin{align}
        d_p(x, z) &\eqcolon \left[\sum_{k=1}^N \vert a_k \vert^p\right]^{1/p} \\
        d_p(y, z) &\eqcolon \left[\sum_{k=1}^N \vert b_k \vert^p\right]^{1/p}.
    \end{align}
    Then, note that $ d_p(x, y) $ (as defined in \cref{eqn:dp-metric}) can be
    written in terms of $ a_k $ and $ b_k $, since $ a_k = x_k - z_k $ and $ b_k
    = y_k - z_k $ for $ k = 1, \ldots, N $:
    \begin{equation}
        d_p(x, y) = \left[\sum_{k=1}^N \vert a_k + b_k \vert^p\right]^{1/p}.
    \end{equation}
    The triangle inequality, as stated in \cref{itm:triangle-inequality},
    requires that
    \begin{equation}
        \left[\sum_{k=1}^N \vert a_k + b_k \vert^p\right]^{1/p} \leq
        \left[\sum_{k=1}^N \vert a_k \vert^p\right]^{1/p} +
        \left[\sum_{k=1}^N \vert b_k \vert^p\right]^{1/p}.
    \end{equation}
    This inequality is equivalent to the well-known Minkowski's Theorem; thus,
    $ d_p $ satisfies the triangle inequality over $ x, y, z \in \mathbb{R}^N $.
\end{theorem}
\section{Lecture II}
\section{Lecture III}
\section{Lecture IV}
\section{Lecture V}
\section{Lecture VI}
\section{Lecture VII}
\section{Lecture VIII}
\section{Lecture IX}
\section{Lecture X}
\section{Lecture XI}
\end{document}

